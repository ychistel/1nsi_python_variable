\documentclass[11pt]{article}

    %\usepackage[breakable]{tcolorbox}
    \usepackage[most]{tcolorbox}
    \usepackage{lmodern} 
    \usepackage{parskip} % Stop auto-indenting (to mimic markdown behaviour)
    
    \usepackage{iftex}
    \ifPDFTeX
    	\usepackage[T1]{fontenc}
    	\usepackage{mathpazo}
    \else
    	\usepackage{fontspec}
    \fi

    % Basic figure setup, for now with no caption control since it's done
    % automatically by Pandoc (which extracts ![](path) syntax from Markdown).
    \usepackage{graphicx}
    % Maintain compatibility with old templates. Remove in nbconvert 6.0
    \let\Oldincludegraphics\includegraphics
    % Ensure that by default, figures have no caption (until we provide a
    % proper Figure object with a Caption API and a way to capture that
    % in the conversion process - todo).
    \usepackage{caption}
    \DeclareCaptionFormat{nocaption}{}
    \captionsetup{format=nocaption,aboveskip=0pt,belowskip=0pt}

    \usepackage{float}
    \floatplacement{figure}{H} % forces figures to be placed at the correct location
    \usepackage{xcolor} % Allow colors to be defined
    \usepackage{enumerate} % Needed for markdown enumerations to work
    \usepackage{geometry} % Used to adjust the document margins
    \usepackage{amsmath} % Equations
    \usepackage{amssymb} % Equations
    \usepackage{textcomp} % defines textquotesingle
    % Hack from http://tex.stackexchange.com/a/47451/13684:
    \AtBeginDocument{%
        \def\PYZsq{\textquotesingle}% Upright quotes in Pygmentized code
    }
    \usepackage{upquote} % Upright quotes for verbatim code
    \usepackage{eurosym} % defines \euro
    \usepackage[mathletters]{ucs} % Extended unicode (utf-8) support
    \usepackage{fancyvrb} % verbatim replacement that allows latex
    \usepackage{grffile} % extends the file name processing of package graphics 
                         % to support a larger range
    \makeatletter % fix for old versions of grffile with XeLaTeX
    \@ifpackagelater{grffile}{2019/11/01}
    {
      % Do nothing on new versions
    }
    {
      \def\Gread@@xetex#1{%
        \IfFileExists{"\Gin@base".bb}%
        {\Gread@eps{\Gin@base.bb}}%
        {\Gread@@xetex@aux#1}%
      }
    }
    \makeatother
    \usepackage[Export]{adjustbox} % Used to constrain images to a maximum size
    \adjustboxset{max size={0.9\linewidth}{0.9\paperheight}}

    % The hyperref package gives us a pdf with properly built
    % internal navigation ('pdf bookmarks' for the table of contents,
    % internal cross-reference links, web links for URLs, etc.)
    \usepackage{hyperref}
    % The default LaTeX title has an obnoxious amount of whitespace. By default,
    % titling removes some of it. It also provides customization options.
    \usepackage{titling}
    \usepackage{longtable} % longtable support required by pandoc >1.10
    \usepackage{booktabs}  % table support for pandoc > 1.12.2
    \usepackage[inline]{enumitem} % IRkernel/repr support (it uses the enumerate* environment)
    \usepackage[normalem]{ulem} % ulem is needed to support strikethroughs (\sout)
                                % normalem makes italics be italics, not underlines
    \usepackage{mathrsfs}
    
\usepackage{fancyhdr}
\usepackage{lastpage}
\usepackage{cclicenses}

    % Colors for the hyperref package
    \definecolor{urlcolor}{rgb}{0,.145,.698}
    \definecolor{linkcolor}{rgb}{.71,0.21,0.01}
    \definecolor{citecolor}{rgb}{.12,.54,.11}

    % ANSI colors
    \definecolor{ansi-black}{HTML}{3E424D}
    \definecolor{ansi-black-intense}{HTML}{282C36}
    \definecolor{ansi-red}{HTML}{E75C58}
    \definecolor{ansi-red-intense}{HTML}{B22B31}
    \definecolor{ansi-green}{HTML}{00A250}
    \definecolor{ansi-green-intense}{HTML}{007427}
    \definecolor{ansi-yellow}{HTML}{DDB62B}
    \definecolor{ansi-yellow-intense}{HTML}{B27D12}
    \definecolor{ansi-blue}{HTML}{208FFB}
    \definecolor{ansi-blue-intense}{HTML}{0065CA}
    \definecolor{ansi-magenta}{HTML}{D160C4}
    \definecolor{ansi-magenta-intense}{HTML}{A03196}
    \definecolor{ansi-cyan}{HTML}{60C6C8}
    \definecolor{ansi-cyan-intense}{HTML}{258F8F}
    \definecolor{ansi-white}{HTML}{C5C1B4}
    \definecolor{ansi-white-intense}{HTML}{A1A6B2}
    \definecolor{ansi-default-inverse-fg}{HTML}{FFFFFF}
    \definecolor{ansi-default-inverse-bg}{HTML}{000000}

    % common color for the border for error outputs.
    \definecolor{outerrorbackground}{HTML}{FFDFDF}

    % commands and environments needed by pandoc snippets
    % extracted from the output of `pandoc -s`
    \providecommand{\tightlist}{%
      \setlength{\itemsep}{0pt}\setlength{\parskip}{0pt}}
    \DefineVerbatimEnvironment{Highlighting}{Verbatim}{commandchars=\\\{\}}
    % Add ',fontsize=\small' for more characters per line
    \newenvironment{Shaded}{}{}
    \newcommand{\KeywordTok}[1]{\textcolor[rgb]{0.00,0.44,0.13}{\textbf{{#1}}}}
    \newcommand{\DataTypeTok}[1]{\textcolor[rgb]{0.56,0.13,0.00}{{#1}}}
    \newcommand{\DecValTok}[1]{\textcolor[rgb]{0.25,0.63,0.44}{{#1}}}
    \newcommand{\BaseNTok}[1]{\textcolor[rgb]{0.25,0.63,0.44}{{#1}}}
    \newcommand{\FloatTok}[1]{\textcolor[rgb]{0.25,0.63,0.44}{{#1}}}
    \newcommand{\CharTok}[1]{\textcolor[rgb]{0.25,0.44,0.63}{{#1}}}
    \newcommand{\StringTok}[1]{\textcolor[rgb]{0.25,0.44,0.63}{{#1}}}
    \newcommand{\CommentTok}[1]{\textcolor[rgb]{0.38,0.63,0.69}{\textit{{#1}}}}
    \newcommand{\OtherTok}[1]{\textcolor[rgb]{0.00,0.44,0.13}{{#1}}}
    \newcommand{\AlertTok}[1]{\textcolor[rgb]{1.00,0.00,0.00}{\textbf{{#1}}}}
    \newcommand{\FunctionTok}[1]{\textcolor[rgb]{0.02,0.16,0.49}{{#1}}}
    \newcommand{\RegionMarkerTok}[1]{{#1}}
    \newcommand{\ErrorTok}[1]{\textcolor[rgb]{1.00,0.00,0.00}{\textbf{{#1}}}}
    \newcommand{\NormalTok}[1]{{#1}}
    
    % Additional commands for more recent versions of Pandoc
    \newcommand{\ConstantTok}[1]{\textcolor[rgb]{0.53,0.00,0.00}{{#1}}}
    \newcommand{\SpecialCharTok}[1]{\textcolor[rgb]{0.25,0.44,0.63}{{#1}}}
    \newcommand{\VerbatimStringTok}[1]{\textcolor[rgb]{0.25,0.44,0.63}{{#1}}}
    \newcommand{\SpecialStringTok}[1]{\textcolor[rgb]{0.73,0.40,0.53}{{#1}}}
    \newcommand{\ImportTok}[1]{{#1}}
    \newcommand{\DocumentationTok}[1]{\textcolor[rgb]{0.73,0.13,0.13}{\textit{{#1}}}}
    \newcommand{\AnnotationTok}[1]{\textcolor[rgb]{0.38,0.63,0.69}{\textbf{\textit{{#1}}}}}
    \newcommand{\CommentVarTok}[1]{\textcolor[rgb]{0.38,0.63,0.69}{\textbf{\textit{{#1}}}}}
    \newcommand{\VariableTok}[1]{\textcolor[rgb]{0.10,0.09,0.49}{{#1}}}
    \newcommand{\ControlFlowTok}[1]{\textcolor[rgb]{0.00,0.44,0.13}{\textbf{{#1}}}}
    \newcommand{\OperatorTok}[1]{\textcolor[rgb]{0.40,0.40,0.40}{{#1}}}
    \newcommand{\BuiltInTok}[1]{{#1}}
    \newcommand{\ExtensionTok}[1]{{#1}}
    \newcommand{\PreprocessorTok}[1]{\textcolor[rgb]{0.74,0.48,0.00}{{#1}}}
    \newcommand{\AttributeTok}[1]{\textcolor[rgb]{0.49,0.56,0.16}{{#1}}}
    \newcommand{\InformationTok}[1]{\textcolor[rgb]{0.38,0.63,0.69}{\textbf{\textit{{#1}}}}}
    \newcommand{\WarningTok}[1]{\textcolor[rgb]{0.38,0.63,0.69}{\textbf{\textit{{#1}}}}}
    
    % Slightly bigger margins than the latex defaults
    
    \geometry{verbose,tmargin=0.7in,bmargin=0.7in,lmargin=0.7in,rmargin=0.7in}
        
    % Define a nice break command that doesn't care if a line doesn't already
    % exist.
    \def\br{\hspace*{\fill} \\* }
    % Math Jax compatibility definitions
    \def\gt{>}
    \def\lt{<}
    \let\Oldtex\TeX
    \let\Oldlatex\LaTeX
    \renewcommand{\TeX}{\textrm{\Oldtex}}
    \renewcommand{\LaTeX}{\textrm{\Oldlatex}}
    % Document parameters
    % Document title
    \title{Opérateurs logiques}
      \date{Octobre 2021}  
	%\author{Yannick Chistel}
    
\makeatletter         
\renewcommand\maketitle[1]{
\hrule\medskip
{\raggedright % Note the extra {
\begin{center}
{\Huge \bfseries \sffamily #1 }\\[4ex] 
%{\Large  \@author}\\[2ex] 
%\@date\\[4ex]
\hrule \bigskip
\end{center}}} % Note the extra }
\makeatother    



\pagestyle{fancy}
\fancyhead{}
\renewcommand\headrulewidth{0pt}
\renewcommand\footrulewidth{1pt}
\fancyfoot[L]{YC}
\fancyfoot[C]{\thepage}
\fancyfoot[R]{\cc-\ccby-\ccnc}

% 
%\newtcolorbox{exemple}[2][]{
%    enhanced,
%    size=fbox,sharp corners,
%    colback=white,colframe=black,
%    colbacktitle=black,fonttitle=\bfseries,
%    attach boxed title to top left={yshift=-3mm,yshifttext=-3mm},
%    boxed title style={size=small,left=0pt,right=0pt,sharp corners},title=#2,#1}

\newtcolorbox{remarque}[2][]{colback=red!4!white,
colframe=red!64!black,fonttitle=\bfseries,
colbacktitle=red!64!black,enhanced,
attach boxed title to top left={xshift=4mm,yshift=-2mm},
title=#2,#1}  


\newtcolorbox{exemple}[2][]{colback=blue!4!white,
colframe=blue!64!green,fonttitle=\bfseries,
colbacktitle=blue!64!green,enhanced,
attach boxed title to top left={xshift=4mm,yshift=-2mm},
title=#2,#1}  
    
% Pygments definitions
\makeatletter
\def\PY@reset{\let\PY@it=\relax \let\PY@bf=\relax%
    \let\PY@ul=\relax \let\PY@tc=\relax%
    \let\PY@bc=\relax \let\PY@ff=\relax}
\def\PY@tok#1{\csname PY@tok@#1\endcsname}
\def\PY@toks#1+{\ifx\relax#1\empty\else%
    \PY@tok{#1}\expandafter\PY@toks\fi}
\def\PY@do#1{\PY@bc{\PY@tc{\PY@ul{%
    \PY@it{\PY@bf{\PY@ff{#1}}}}}}}
\def\PY#1#2{\PY@reset\PY@toks#1+\relax+\PY@do{#2}}

\@namedef{PY@tok@w}{\def\PY@tc##1{\textcolor[rgb]{0.73,0.73,0.73}{##1}}}
\@namedef{PY@tok@c}{\let\PY@it=\textit\def\PY@tc##1{\textcolor[rgb]{0.25,0.50,0.50}{##1}}}
\@namedef{PY@tok@cp}{\def\PY@tc##1{\textcolor[rgb]{0.74,0.48,0.00}{##1}}}
\@namedef{PY@tok@k}{\let\PY@bf=\textbf\def\PY@tc##1{\textcolor[rgb]{0.00,0.50,0.00}{##1}}}
\@namedef{PY@tok@kp}{\def\PY@tc##1{\textcolor[rgb]{0.00,0.50,0.00}{##1}}}
\@namedef{PY@tok@kt}{\def\PY@tc##1{\textcolor[rgb]{0.69,0.00,0.25}{##1}}}
\@namedef{PY@tok@o}{\def\PY@tc##1{\textcolor[rgb]{0.40,0.40,0.40}{##1}}}
\@namedef{PY@tok@ow}{\let\PY@bf=\textbf\def\PY@tc##1{\textcolor[rgb]{0.67,0.13,1.00}{##1}}}
\@namedef{PY@tok@nb}{\def\PY@tc##1{\textcolor[rgb]{0.00,0.50,0.00}{##1}}}
\@namedef{PY@tok@nf}{\def\PY@tc##1{\textcolor[rgb]{0.00,0.00,1.00}{##1}}}
\@namedef{PY@tok@nc}{\let\PY@bf=\textbf\def\PY@tc##1{\textcolor[rgb]{0.00,0.00,1.00}{##1}}}
\@namedef{PY@tok@nn}{\let\PY@bf=\textbf\def\PY@tc##1{\textcolor[rgb]{0.00,0.00,1.00}{##1}}}
\@namedef{PY@tok@ne}{\let\PY@bf=\textbf\def\PY@tc##1{\textcolor[rgb]{0.82,0.25,0.23}{##1}}}
\@namedef{PY@tok@nv}{\def\PY@tc##1{\textcolor[rgb]{0.10,0.09,0.49}{##1}}}
\@namedef{PY@tok@no}{\def\PY@tc##1{\textcolor[rgb]{0.53,0.00,0.00}{##1}}}
\@namedef{PY@tok@nl}{\def\PY@tc##1{\textcolor[rgb]{0.63,0.63,0.00}{##1}}}
\@namedef{PY@tok@ni}{\let\PY@bf=\textbf\def\PY@tc##1{\textcolor[rgb]{0.60,0.60,0.60}{##1}}}
\@namedef{PY@tok@na}{\def\PY@tc##1{\textcolor[rgb]{0.49,0.56,0.16}{##1}}}
\@namedef{PY@tok@nt}{\let\PY@bf=\textbf\def\PY@tc##1{\textcolor[rgb]{0.00,0.50,0.00}{##1}}}
\@namedef{PY@tok@nd}{\def\PY@tc##1{\textcolor[rgb]{0.67,0.13,1.00}{##1}}}
\@namedef{PY@tok@s}{\def\PY@tc##1{\textcolor[rgb]{0.73,0.13,0.13}{##1}}}
\@namedef{PY@tok@sd}{\let\PY@it=\textit\def\PY@tc##1{\textcolor[rgb]{0.73,0.13,0.13}{##1}}}
\@namedef{PY@tok@si}{\let\PY@bf=\textbf\def\PY@tc##1{\textcolor[rgb]{0.73,0.40,0.53}{##1}}}
\@namedef{PY@tok@se}{\let\PY@bf=\textbf\def\PY@tc##1{\textcolor[rgb]{0.73,0.40,0.13}{##1}}}
\@namedef{PY@tok@sr}{\def\PY@tc##1{\textcolor[rgb]{0.73,0.40,0.53}{##1}}}
\@namedef{PY@tok@ss}{\def\PY@tc##1{\textcolor[rgb]{0.10,0.09,0.49}{##1}}}
\@namedef{PY@tok@sx}{\def\PY@tc##1{\textcolor[rgb]{0.00,0.50,0.00}{##1}}}
\@namedef{PY@tok@m}{\def\PY@tc##1{\textcolor[rgb]{0.40,0.40,0.40}{##1}}}
\@namedef{PY@tok@gh}{\let\PY@bf=\textbf\def\PY@tc##1{\textcolor[rgb]{0.00,0.00,0.50}{##1}}}
\@namedef{PY@tok@gu}{\let\PY@bf=\textbf\def\PY@tc##1{\textcolor[rgb]{0.50,0.00,0.50}{##1}}}
\@namedef{PY@tok@gd}{\def\PY@tc##1{\textcolor[rgb]{0.63,0.00,0.00}{##1}}}
\@namedef{PY@tok@gi}{\def\PY@tc##1{\textcolor[rgb]{0.00,0.63,0.00}{##1}}}
\@namedef{PY@tok@gr}{\def\PY@tc##1{\textcolor[rgb]{1.00,0.00,0.00}{##1}}}
\@namedef{PY@tok@ge}{\let\PY@it=\textit}
\@namedef{PY@tok@gs}{\let\PY@bf=\textbf}
\@namedef{PY@tok@gp}{\let\PY@bf=\textbf\def\PY@tc##1{\textcolor[rgb]{0.00,0.00,0.50}{##1}}}
\@namedef{PY@tok@go}{\def\PY@tc##1{\textcolor[rgb]{0.53,0.53,0.53}{##1}}}
\@namedef{PY@tok@gt}{\def\PY@tc##1{\textcolor[rgb]{0.00,0.27,0.87}{##1}}}
\@namedef{PY@tok@err}{\def\PY@bc##1{{\setlength{\fboxsep}{\string -\fboxrule}\fcolorbox[rgb]{1.00,0.00,0.00}{1,1,1}{\strut ##1}}}}
\@namedef{PY@tok@kc}{\let\PY@bf=\textbf\def\PY@tc##1{\textcolor[rgb]{0.00,0.50,0.00}{##1}}}
\@namedef{PY@tok@kd}{\let\PY@bf=\textbf\def\PY@tc##1{\textcolor[rgb]{0.00,0.50,0.00}{##1}}}
\@namedef{PY@tok@kn}{\let\PY@bf=\textbf\def\PY@tc##1{\textcolor[rgb]{0.00,0.50,0.00}{##1}}}
\@namedef{PY@tok@kr}{\let\PY@bf=\textbf\def\PY@tc##1{\textcolor[rgb]{0.00,0.50,0.00}{##1}}}
\@namedef{PY@tok@bp}{\def\PY@tc##1{\textcolor[rgb]{0.00,0.50,0.00}{##1}}}
\@namedef{PY@tok@fm}{\def\PY@tc##1{\textcolor[rgb]{0.00,0.00,1.00}{##1}}}
\@namedef{PY@tok@vc}{\def\PY@tc##1{\textcolor[rgb]{0.10,0.09,0.49}{##1}}}
\@namedef{PY@tok@vg}{\def\PY@tc##1{\textcolor[rgb]{0.10,0.09,0.49}{##1}}}
\@namedef{PY@tok@vi}{\def\PY@tc##1{\textcolor[rgb]{0.10,0.09,0.49}{##1}}}
\@namedef{PY@tok@vm}{\def\PY@tc##1{\textcolor[rgb]{0.10,0.09,0.49}{##1}}}
\@namedef{PY@tok@sa}{\def\PY@tc##1{\textcolor[rgb]{0.73,0.13,0.13}{##1}}}
\@namedef{PY@tok@sb}{\def\PY@tc##1{\textcolor[rgb]{0.73,0.13,0.13}{##1}}}
\@namedef{PY@tok@sc}{\def\PY@tc##1{\textcolor[rgb]{0.73,0.13,0.13}{##1}}}
\@namedef{PY@tok@dl}{\def\PY@tc##1{\textcolor[rgb]{0.73,0.13,0.13}{##1}}}
\@namedef{PY@tok@s2}{\def\PY@tc##1{\textcolor[rgb]{0.73,0.13,0.13}{##1}}}
\@namedef{PY@tok@sh}{\def\PY@tc##1{\textcolor[rgb]{0.73,0.13,0.13}{##1}}}
\@namedef{PY@tok@s1}{\def\PY@tc##1{\textcolor[rgb]{0.73,0.13,0.13}{##1}}}
\@namedef{PY@tok@mb}{\def\PY@tc##1{\textcolor[rgb]{0.40,0.40,0.40}{##1}}}
\@namedef{PY@tok@mf}{\def\PY@tc##1{\textcolor[rgb]{0.40,0.40,0.40}{##1}}}
\@namedef{PY@tok@mh}{\def\PY@tc##1{\textcolor[rgb]{0.40,0.40,0.40}{##1}}}
\@namedef{PY@tok@mi}{\def\PY@tc##1{\textcolor[rgb]{0.40,0.40,0.40}{##1}}}
\@namedef{PY@tok@il}{\def\PY@tc##1{\textcolor[rgb]{0.40,0.40,0.40}{##1}}}
\@namedef{PY@tok@mo}{\def\PY@tc##1{\textcolor[rgb]{0.40,0.40,0.40}{##1}}}
\@namedef{PY@tok@ch}{\let\PY@it=\textit\def\PY@tc##1{\textcolor[rgb]{0.25,0.50,0.50}{##1}}}
\@namedef{PY@tok@cm}{\let\PY@it=\textit\def\PY@tc##1{\textcolor[rgb]{0.25,0.50,0.50}{##1}}}
\@namedef{PY@tok@cpf}{\let\PY@it=\textit\def\PY@tc##1{\textcolor[rgb]{0.25,0.50,0.50}{##1}}}
\@namedef{PY@tok@c1}{\let\PY@it=\textit\def\PY@tc##1{\textcolor[rgb]{0.25,0.50,0.50}{##1}}}
\@namedef{PY@tok@cs}{\let\PY@it=\textit\def\PY@tc##1{\textcolor[rgb]{0.25,0.50,0.50}{##1}}}

\def\PYZbs{\char`\\}
\def\PYZus{\char`\_}
\def\PYZob{\char`\{}
\def\PYZcb{\char`\}}
\def\PYZca{\char`\^}
\def\PYZam{\char`\&}
\def\PYZlt{\char`\<}
\def\PYZgt{\char`\>}
\def\PYZsh{\char`\#}
\def\PYZpc{\char`\%}
\def\PYZdl{\char`\$}
\def\PYZhy{\char`\-}
\def\PYZsq{\char`\'}
\def\PYZdq{\char`\"}
\def\PYZti{\char`\~}
% for compatibility with earlier versions
\def\PYZat{@}
\def\PYZlb{[}
\def\PYZrb{]}
\makeatother


    % For linebreaks inside Verbatim environment from package fancyvrb. 
    \makeatletter
        \newbox\Wrappedcontinuationbox 
        \newbox\Wrappedvisiblespacebox 
        \newcommand*\Wrappedvisiblespace {\textcolor{red}{\textvisiblespace}} 
        \newcommand*\Wrappedcontinuationsymbol {\textcolor{red}{\llap{\tiny$\m@th\hookrightarrow$}}} 
        \newcommand*\Wrappedcontinuationindent {3ex } 
        \newcommand*\Wrappedafterbreak {\kern\Wrappedcontinuationindent\copy\Wrappedcontinuationbox} 
        % Take advantage of the already applied Pygments mark-up to insert 
        % potential linebreaks for TeX processing. 
        %        {, <, #, %, $, ' and ": go to next line. 
        %        _, }, ^, &, >, - and ~: stay at end of broken line. 
        % Use of \textquotesingle for straight quote. 
        \newcommand*\Wrappedbreaksatspecials {% 
            \def\PYGZus{\discretionary{\char`\_}{\Wrappedafterbreak}{\char`\_}}% 
            \def\PYGZob{\discretionary{}{\Wrappedafterbreak\char`\{}{\char`\{}}% 
            \def\PYGZcb{\discretionary{\char`\}}{\Wrappedafterbreak}{\char`\}}}% 
            \def\PYGZca{\discretionary{\char`\^}{\Wrappedafterbreak}{\char`\^}}% 
            \def\PYGZam{\discretionary{\char`\&}{\Wrappedafterbreak}{\char`\&}}% 
            \def\PYGZlt{\discretionary{}{\Wrappedafterbreak\char`\<}{\char`\<}}% 
            \def\PYGZgt{\discretionary{\char`\>}{\Wrappedafterbreak}{\char`\>}}% 
            \def\PYGZsh{\discretionary{}{\Wrappedafterbreak\char`\#}{\char`\#}}% 
            \def\PYGZpc{\discretionary{}{\Wrappedafterbreak\char`\%}{\char`\%}}% 
            \def\PYGZdl{\discretionary{}{\Wrappedafterbreak\char`\$}{\char`\$}}% 
            \def\PYGZhy{\discretionary{\char`\-}{\Wrappedafterbreak}{\char`\-}}% 
            \def\PYGZsq{\discretionary{}{\Wrappedafterbreak\textquotesingle}{\textquotesingle}}% 
            \def\PYGZdq{\discretionary{}{\Wrappedafterbreak\char`\"}{\char`\"}}% 
            \def\PYGZti{\discretionary{\char`\~}{\Wrappedafterbreak}{\char`\~}}% 
        } 
        % Some characters . , ; ? ! / are not pygmentized. 
        % This macro makes them "active" and they will insert potential linebreaks 
        \newcommand*\Wrappedbreaksatpunct {% 
            \lccode`\~`\.\lowercase{\def~}{\discretionary{\hbox{\char`\.}}{\Wrappedafterbreak}{\hbox{\char`\.}}}% 
            \lccode`\~`\,\lowercase{\def~}{\discretionary{\hbox{\char`\,}}{\Wrappedafterbreak}{\hbox{\char`\,}}}% 
            \lccode`\~`\;\lowercase{\def~}{\discretionary{\hbox{\char`\;}}{\Wrappedafterbreak}{\hbox{\char`\;}}}% 
            \lccode`\~`\:\lowercase{\def~}{\discretionary{\hbox{\char`\:}}{\Wrappedafterbreak}{\hbox{\char`\:}}}% 
            \lccode`\~`\?\lowercase{\def~}{\discretionary{\hbox{\char`\?}}{\Wrappedafterbreak}{\hbox{\char`\?}}}% 
            \lccode`\~`\!\lowercase{\def~}{\discretionary{\hbox{\char`\!}}{\Wrappedafterbreak}{\hbox{\char`\!}}}% 
            \lccode`\~`\/\lowercase{\def~}{\discretionary{\hbox{\char`\/}}{\Wrappedafterbreak}{\hbox{\char`\/}}}% 
            \catcode`\.\active
            \catcode`\,\active 
            \catcode`\;\active
            \catcode`\:\active
            \catcode`\?\active
            \catcode`\!\active
            \catcode`\/\active 
            \lccode`\~`\~ 	
        }
    \makeatother

    \let\OriginalVerbatim=\Verbatim
    \makeatletter
    \renewcommand{\Verbatim}[1][1]{%
        %\parskip\z@skip
        \sbox\Wrappedcontinuationbox {\Wrappedcontinuationsymbol}%
        \sbox\Wrappedvisiblespacebox {\FV@SetupFont\Wrappedvisiblespace}%
        \def\FancyVerbFormatLine ##1{\hsize\linewidth
            \vtop{\raggedright\hyphenpenalty\z@\exhyphenpenalty\z@
                \doublehyphendemerits\z@\finalhyphendemerits\z@
                \strut ##1\strut}%
        }%
        % If the linebreak is at a space, the latter will be displayed as visible
        % space at end of first line, and a continuation symbol starts next line.
        % Stretch/shrink are however usually zero for typewriter font.
        \def\FV@Space {%
            \nobreak\hskip\z@ plus\fontdimen3\font minus\fontdimen4\font
            \discretionary{\copy\Wrappedvisiblespacebox}{\Wrappedafterbreak}
            {\kern\fontdimen2\font}%
        }%
        
        % Allow breaks at special characters using \PYG... macros.
        \Wrappedbreaksatspecials
        % Breaks at punctuation characters . , ; ? ! and / need catcode=\active 	
        \OriginalVerbatim[#1,codes*=\Wrappedbreaksatpunct]%
    }
    \makeatother

    % Exact colors from NB
    \definecolor{incolor}{HTML}{303F9F}
    \definecolor{outcolor}{HTML}{D84315}
    \definecolor{cellborder}{HTML}{CFCFCF}
    \definecolor{cellbackground}{HTML}{F7F7F7}
    
    % prompt
    \makeatletter
    \newcommand{\boxspacing}{\kern\kvtcb@left@rule\kern\kvtcb@boxsep}
    \makeatother
    \newcommand{\prompt}[4]{
        {\ttfamily\llap{{\color{#2}[#3]:\hspace{3pt}#4}}\vspace{-\baselineskip}}
    }
    

    
    % Prevent overflowing lines due to hard-to-break entities
    \sloppy 
    % Setup hyperref package
    \hypersetup{
      breaklinks=true,  % so long urls are correctly broken across lines
      colorlinks=true,
      urlcolor=urlcolor,
      linkcolor=linkcolor,
      citecolor=citecolor,
      }

    
\setcounter{subsection}{0}

\begin{document}
    
    \maketitle{Exercices : base du langage Python}
    
    
%
%    
%    \hypertarget{exercices-base-du-langage-python}{%
%\section{Exercices : base du langage
%Python}\label{exercices-base-du-langage-python}}


    \hypertarget{exercice}{%
\section{Exercice}\label{exercice}}

\begin{enumerate}
\def\labelenumi{\arabic{enumi}.}
\tightlist
\item
  Déclarer dans l'interpréteur Python, 2 variables \(a\) et \(b\) en
  leur affectant respectivement les valeurs \(15\) et \(7\).
\item
  Calculer \texttt{a//b} puis \texttt{b//a}. Que signifient les valeurs
  affichées?
\item
  Calculer \texttt{a\%b} et \texttt{b\%a}. Que signifient les valeurs
  affichées?
\item
  Calculer \texttt{a/2+b/3}. Quel est le type numérique obtenu ?
\item
  Calculer \texttt{a**4-b**3}. Quel est le calcul effectué ?
\end{enumerate}

    \hypertarget{exercice}{%
\section{Exercice}\label{exercice}}

\begin{enumerate}
\def\labelenumi{\arabic{enumi}.}
\tightlist
\item
  Quelle est la valeur de la variable \(a\) si on saisit les lignes
  suivantes dans l'interpréteur :
\end{enumerate}

\begin{Shaded}
\begin{Highlighting}[]
\OperatorTok{\textgreater{}\textgreater{}\textgreater{}}\NormalTok{ a}\OperatorTok{=}\DecValTok{3}
\OperatorTok{\textgreater{}\textgreater{}\textgreater{}}\NormalTok{ a}\OperatorTok{=}\DecValTok{4}    
\OperatorTok{\textgreater{}\textgreater{}\textgreater{}}\NormalTok{ a}\OperatorTok{=}\NormalTok{a}\OperatorTok{+}\DecValTok{2}    
\OperatorTok{\textgreater{}\textgreater{}\textgreater{}}\NormalTok{ a     }
\end{Highlighting}
\end{Shaded}

\begin{enumerate}
\def\labelenumi{\arabic{enumi}.}
\setcounter{enumi}{1}
\tightlist
\item
  Refaire de même avec :
\end{enumerate}

\begin{Shaded}
\begin{Highlighting}[]
\OperatorTok{\textgreater{}\textgreater{}\textgreater{}}\NormalTok{ a}\OperatorTok{=}\DecValTok{2}
\OperatorTok{\textgreater{}\textgreater{}\textgreater{}}\NormalTok{ b}\OperatorTok{=}\NormalTok{a}\OperatorTok{*}\NormalTok{a}
\OperatorTok{\textgreater{}\textgreater{}\textgreater{}}\NormalTok{ b}\OperatorTok{=}\NormalTok{a}\OperatorTok{*}\NormalTok{b}
\OperatorTok{\textgreater{}\textgreater{}\textgreater{}}\NormalTok{ b}\OperatorTok{=}\NormalTok{b}\OperatorTok{*}\NormalTok{b}
\OperatorTok{\textgreater{}\textgreater{}\textgreater{}}\NormalTok{ b}
\end{Highlighting}
\end{Shaded}

\begin{enumerate}
\def\labelenumi{\arabic{enumi}.}
\setcounter{enumi}{2}
\tightlist
\item
  Vérifier vos réponses en saisissant ces lignes dans l'interpréteur.
\end{enumerate}

    \hypertarget{exercice}{%
\section{Exercice}\label{exercice}}

On donne le programme suivant:

\begin{Shaded}
\begin{Highlighting}[]
\NormalTok{a }\OperatorTok{=} \BuiltInTok{int}\NormalTok{(}\BuiltInTok{input}\NormalTok{(}\StringTok{"Valeur de a:"}\NormalTok{))}
\NormalTok{b }\OperatorTok{=} \BuiltInTok{int}\NormalTok{(}\BuiltInTok{input}\NormalTok{(}\StringTok{"Valeur de b:"}\NormalTok{))}
\NormalTok{a,b }\OperatorTok{=}\NormalTok{ b,a}
\BuiltInTok{print}\NormalTok{(}\StringTok{\textquotesingle{}valeur de a:\textquotesingle{}}\NormalTok{,a)}
\BuiltInTok{print}\NormalTok{(}\StringTok{\textquotesingle{}valeur de b:\textquotesingle{}}\NormalTok{,b)}
\end{Highlighting}
\end{Shaded}

\begin{enumerate}
\def\labelenumi{\arabic{enumi}.}
\tightlist
\item
  Que fait ce programme si on saisit la valeur 5 pour \texttt{a} et 3
  pour \texttt{b} ?
\item
  Saisir dans l'éditeur le programme suivant et l'enregistrer sous le
  nom \textbf{vice-versa.py}.
\item
  Exécuter ce programme et saisir les valeurs 5 et 3.
\item
  Supprimer la ligne \texttt{a,b\ =\ b,a} de ce programme puis écrire
  les instructions en Python qui permettent de faire la même action en
  utilisant 3 variables \texttt{a}, \texttt{b} et \texttt{c}.
\end{enumerate}

    \hypertarget{exercice}{%
\section{Exercice}\label{exercice}}

Depuis l'interpréteur, accéder à l'éditeur python, puis écrire un
programme qui :

\begin{enumerate}
\def\labelenumi{\arabic{enumi}.}
\tightlist
\item
  demande la saisie d'une valeur pour la variable \(a\).
\item
  demande la saisie d'une valeur pour la variable \(b\).
\item
  calcule et affiche la somme des nombres \(a\) et \(b\).
\item
  Que remarquez-vous ? Corriger si nécessaire.
\end{enumerate}

    \hypertarget{exercice}{%
\section{Exercice}\label{exercice}}

\begin{enumerate}
\def\labelenumi{\arabic{enumi}.}
\tightlist
\item
  Dans l'interpréteur, saisir :
\end{enumerate}

\begin{enumerate}
\def\labelenumi{\alph{enumi})}

\item
  \texttt{int(\textquotesingle{}0110\textquotesingle{},2)}
\item
  \texttt{int(\textquotesingle{}FF\textquotesingle{},16)}
\item
  \texttt{int(\textquotesingle{}10011001\textquotesingle{},\ 2)}
\end{enumerate}

\begin{enumerate}
\def\labelenumi{\arabic{enumi}.}
\setcounter{enumi}{1}
\tightlist
\item
  Que semble réaliser la fonction python \texttt{int(chaine,\ base)} ?
\item
  Écrire un programme qui demande la saisie d'une base (entre 2 et 36)
  et un nombre dans cette base et affiche ce nombre en base 10.
\end{enumerate}

    \hypertarget{exercice}{%
\section{Exercice}\label{exercice}}

\begin{enumerate}
\def\labelenumi{\arabic{enumi}.}
\tightlist
\item
  Les tests suivants sont-ils vrais ou faux ?
\end{enumerate}

\begin{enumerate}
\def\labelenumi{\alph{enumi})}

\item
  \texttt{5\ \textgreater{}\ 0}
\item
  \texttt{25\ \textgreater{}\ 0\ and\ 25\ \textless{}\ 100}
\item
  \texttt{36\ \textgreater{}\ 0\ or\ 36\ \textless{}\ 100}
\item
  \texttt{10\ \textless{}\ 0\ or\ 10\ \textgreater{}\ 0}
\item
  \texttt{-1\ \textless{}\ 0\ and\ -1\ \textgreater{}\ 1}
\item
  \texttt{not\ -1\ \textless{}\ 0}
\item
  \texttt{not(1\textgreater{}0\ or\ 0\textgreater{}1)}
\item
  \texttt{not(1\textgreater{}0\ and\ 0\textgreater{}1)}
\item
  \texttt{3\ \%\ 2\ ==\ 0}
\item
  \texttt{7\ //\ 2\ ==\ 3}
\item
  \texttt{0.1\ +\ 0.2\ ==\ 0.3}
\end{enumerate}

\begin{enumerate}
\def\labelenumi{\arabic{enumi}.}
\setcounter{enumi}{1}
\tightlist
\item
  Vérifiez vos réponses en les saisissant dans l'interpréteur Python
\end{enumerate}

    \hypertarget{exercice}{%
\section{Exercice}\label{exercice}}

On donne le programme incomplet suivant:

\begin{Shaded}
\begin{Highlighting}[]
\NormalTok{n }\OperatorTok{=} \BuiltInTok{int}\NormalTok{(}\BuiltInTok{input}\NormalTok{(}\StringTok{"Valeur de n:"}\NormalTok{))}
\ControlFlowTok{if}\NormalTok{ ... :}
    \BuiltInTok{print}\NormalTok{(}\StringTok{\textquotesingle{}le nombre n est pair\textquotesingle{}}\NormalTok{)}
\ControlFlowTok{else}\NormalTok{:}
    \BuiltInTok{print}\NormalTok{(}\StringTok{\textquotesingle{}le nombre n est impair\textquotesingle{}}\NormalTok{)}
\end{Highlighting}
\end{Shaded}

\begin{enumerate}
\def\labelenumi{\arabic{enumi}.}
\tightlist
\item
  Compléter ce programme pour qu'il affiche la bonne réponse dans tous
  les cas.
\item
  Saisir ce programme dans l'éditeur Python, l'enregistrer sous le nom
  \textbf{parite.py} puis l'exécuter.
\end{enumerate}

    \hypertarget{exercice}{%
\section{Exercice}\label{exercice}}

\begin{enumerate}
\def\labelenumi{\arabic{enumi}.}
\tightlist
\item
  Créer un nouveau fichier Python dans l'éditeur et l'enregistrer sous
  le nom \textbf{multiple.py}.
\item
  Reprendre le programme de l'exercice précédent et le modifier pour
  vérifier que le nombre saisi est un multiple de 3.
\item
  Modifier le programme précédent pour dire si le nombre saisi est un
  multiple de 5.
\item
  Modifier votre programme pour dire si le nombre saisi est :
\end{enumerate}

\begin{itemize}
\tightlist
\item
  un multiple de 3 uniquement;
\item
  un multiple de 5 uniquement;
\item
  un multiple de 3 et 5.
\end{itemize}

    \hypertarget{exercice}{%
\section{Exercice}\label{exercice}}

\begin{enumerate}
\def\labelenumi{\arabic{enumi}.}
\tightlist
\item
  Créer un nouveau fichier Python dans l'éditeur et l'enregistrer sous
  le nom \textbf{multiple-a-b.py}.
\item
  Écrire un programme qui :
\end{enumerate}

\begin{itemize}
\tightlist
\item
  demande la saisie de deux nombres entiers \texttt{a} et \texttt{b};
\item
  affiche si l'un est un multiple de l'autre.
\item
  on traitera les deux cas \texttt{a} est un multiple de \texttt{b} ou
  \texttt{b} est un multiple de \texttt{a}.
\end{itemize}

    \hypertarget{exercice}{%
\section{Exercice}\label{exercice}}

\begin{enumerate}
\def\labelenumi{\arabic{enumi}.}
\tightlist
\item
  Créer un nouveau fichier Python dans l'éditeur et l'enregistrer sous
  le nom \textbf{identite.py}.
\item
  Écrire un programme en Python qui :
\end{enumerate}

\begin{itemize}
\tightlist
\item
  demande de saisir le nom dans une variable \textbf{nom}.
\item
  demande de saisir le prénom dans une variable \textbf{prenom}.
\item
  demande de saisir l'âge dans la variable \textbf{age}.
\item
  affiche le message ``Vous vous appelez prenom nom et vous avez age
  ans.'' (en remplaçant les variables nom, prénom et age par leurs
  valeurs).
\end{itemize}

\begin{enumerate}
\def\labelenumi{\arabic{enumi}.}
\setcounter{enumi}{2}
\tightlist
\item
  Modifier votre programme avec un test qui permet de dire si la
  personne est plus jeune, plus vielle ou a le même âge que vous.
\end{enumerate}

    \hypertarget{exercice}{%
\section{Exercice}\label{exercice}}

Écrire un programme qui demande la saisie des trois longueurs d'un
triangle et affiche sa nature selon les cas.

On rappelle qu'un triangle est iscèle s'il a 2 côtés de même longueur et
qu'il est équilatéral s'il a ses trois cotés de même longueur.

    \hypertarget{exercice}{%
\section{Exercice}\label{exercice}}

Écrire un programme qui demande la saisie d'une année \(n\) et affiche
si elle est bissextile ou non.\\
On rappelle qu'une année est bissextile si elle est un multiple de 4
mais pas un multiple de 100, ou si elle est un multiple de 400.

    \hypertarget{exercice}{%
\section{Exercice}\label{exercice}}

La solution des équations du type \(ax+b=0\) est \(\dfrac{-b}{a}\)
lorsque \(a\) est non nul. Si \(a\) est nul, alors soit il y a infinité
de solutions lorsque \(b=0\), soit aucune solution dans le cas
contraire.

Écrire un programme qui : - demande la saisie des coefficients
\texttt{a} et \texttt{b}. - donne la solution de l'équation \(ax+b=0\)
selon les différents cas de figure.

    \hypertarget{exercice}{%
\section{Exercice}\label{exercice}}

\begin{enumerate}
\def\labelenumi{\arabic{enumi}.}
\tightlist
\item
  Écrire un programme qui demande la saisie de 2 nombres entiers puis
  affiche ces nombres dans l'ordre croissant.
\item
  Écrire un programme qui demande la saisie de 3 nombres entiers puis
  affiche ces nombres dans l'ordre croissant.
\item
  Combien de tests de comparaison faudrait-il pour 4 nombres entiers ?
\end{enumerate}

    \hypertarget{exercice}{%
\section{Exercice}\label{exercice}}

Une fonction polynôme du second degré est de la forme \(ax^2+bx+c\) avec
$ a \neq  0$.\\
On propose de déterminer si le polynôme admet des racines, autrement
dit, si l'équation \(ax^2+bx+c=0\) admet des solutions. Écrire un
programme en python pour:

\begin{enumerate}
\def\labelenumi{\arabic{enumi}.}
\tightlist
\item
  Saisir la valeur de chaque coefficient a, b et c du trinôme.
\item
  Calculer le discriminant dans une variable delta et afficher sa
  valeur.
\item
  Dire si le trinôme a 0, 1 ou 2 racines et en donner les valeurs.
\item
  Tester votre code avec les trinômes \(-x^2+3x-4\), \(-x^2+2x-1\) et
  \(\dfrac{1}{2}x^2+x-7\).
\item
  Modifier votre programme pour refuser la valeur \(a=0\) en précisant
  que ce n'est pas du second degré.
\end{enumerate}
    
\end{document}
